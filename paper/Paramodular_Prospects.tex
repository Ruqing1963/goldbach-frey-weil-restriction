\documentclass[11pt,a4paper]{article}
\usepackage[margin=2.5cm]{geometry}
\usepackage{amsmath,amssymb,amsthm}
\usepackage{booktabs}
\usepackage{graphicx}
\usepackage{hyperref}
\usepackage{float}
\usepackage{url}

\newcommand{\rad}{\operatorname{rad}}
\newcommand{\Jac}{\operatorname{Jac}}
\newcommand{\Cond}{\operatorname{Cond}}
\newcommand{\End}{\operatorname{End}}
\newcommand{\Res}{\operatorname{Res}}
\newcommand{\Tr}{\operatorname{Tr}}
\newcommand{\Frob}{\operatorname{Frob}}
\newcommand{\Aut}{\operatorname{Aut}}
\newcommand{\GSp}{\mathrm{GSp}}
\newcommand{\GL}{\mathrm{GL}}
\newcommand{\USp}{\mathrm{USp}}
\newcommand{\Gm}{\mathbb{G}_{\mathrm{m}}}
\newcommand{\Ga}{\mathbb{G}_{\mathrm{a}}}
\newcommand{\Qbar}{\overline{\mathbb{Q}}}
\newcommand{\Ql}{\mathbb{Q}_\ell}

\newtheorem{theorem}{Theorem}[section]
\newtheorem{proposition}[theorem]{Proposition}
\newtheorem{corollary}[theorem]{Corollary}
\newtheorem{conjecture}[theorem]{Conjecture}
\newtheorem{lemma}[theorem]{Lemma}
\theoremstyle{definition}
\newtheorem{definition}[theorem]{Definition}
\theoremstyle{remark}
\newtheorem{remark}[theorem]{Remark}

\title{Goldbach--Frey Jacobians as Weil Restrictions:\\
  Trace Vanishing, Asai $L$-Functions,\\
  and Modularity via $\mathbb{Q}(i)$}
\author{Ruqing Chen\\[4pt]
  \small GUT Geoservice Inc., Montreal\\
  \small\texttt{ruqing@hotmail.com}}
\date{February 2026}

\begin{document}
\maketitle

\begin{abstract}
For the Goldbach--Frey curve
$C\colon y^2 = x(x^2 - p^2)(x^2 - q^2)$ with $p \ne q$
distinct odd primes, we prove that the Frobenius trace
$a_r = 0$ at every good prime $r \equiv 3 \pmod{4}$.
This ``trace vanishing law'' is the signature of an
induced Galois representation: the involution
$(x,y) \mapsto (-x, iy)$ defined over $\mathbb{Q}(i)$
forces the $\ell$-adic representation of $\Jac(C)$ to
be induced from $G_{\mathbb{Q}(i)}$ to $G_{\mathbb{Q}}$,
so that $\Jac(C)$ is isogenous over~$\mathbb{Q}$ to the
Weil restriction $\Res_{\mathbb{Q}(i)/\mathbb{Q}}(E)$
of an elliptic curve $E/\mathbb{Q}(i)$.
The degree-$4$ $L$-function is therefore an Asai
$L$-function (a Langlands lift from
$\GL_2(\mathbb{Q}(i))$ to $\GSp_4(\mathbb{Q})$), and
the associated Siegel paramodular form is an endoscopic
lift.
We compute explicit local $L$-factors at all bad odd
primes---including a quadratic-residue analysis of node
splitting---and show that modularity follows from the
modularity of $E/\mathbb{Q}(i)$ via automorphic induction.
\end{abstract}


% ====================================================================
\section{Introduction}

The Goldbach--Frey curve
\begin{equation}\label{eq:curve}
  C\colon y^2 = f(x) = x(x^2 - p^2)(x^2 - q^2)
\end{equation}
has the palindromic property $f(-x) = -f(x)$.  This symmetry,
inherited from the additive constraint $p + q = 2N$, has
arithmetic consequences that go beyond the conductor analysis
of Papers~\cite{Chen2026TCV,Chen2026UTS}.

The map
\begin{equation}\label{eq:involution}
  w\colon (x, y) \longmapsto (-x,\, iy)
\end{equation}
is an automorphism of~$C$ of order~$4$, defined over
$\mathbb{Q}(i)$ but not over~$\mathbb{Q}$.
This involution acts on the $\ell$-adic Tate module
$V_\ell(\Jac(C))$ and constrains the Frobenius eigenvalues,
producing a ``trace vanishing law'' at all inert primes.

In this paper, we:
\begin{enumerate}
\item prove that $a_r = 0$ for every good prime
  $r \equiv 3 \pmod{4}$ (Theorem~\ref{thm:vanish});
\item show that $\End_{\mathbb{Q}}(\Jac(C)) = \mathbb{Z}$
  for generic $(p, q)$ (Proposition~\ref{prop:endQ});
\item identify $\Jac(C)$ as a Weil restriction from
  $\mathbb{Q}(i)$ and the $L$-function as an Asai lift
  (Section~\ref{sec:weil});
\item compute refined local $L$-factors at all bad odd primes,
  including a split/non-split node analysis
  (Section~\ref{sec:local});
\item establish modularity via automorphic induction from
  $\GL_2(\mathbb{Q}(i))$
  (Section~\ref{sec:modularity}).
\end{enumerate}


% ====================================================================
\section{The Trace Vanishing Law}\label{sec:vanish}

\begin{theorem}\label{thm:vanish}
  Let $r > 2$ be a prime of good reduction for~$C$, and let
  $a_r = r + 1 - \#C(\mathbb{F}_r)$ be the Frobenius trace.
  If $r \equiv 3 \pmod{4}$, then $a_r = 0$.
\end{theorem}

\begin{proof}
  Over $\mathbb{F}_r$, the number of
  affine points on $C\colon y^2 = xg(x)$ where
  $g(x) = (x^2 - p^2)(x^2 - q^2)$ is
  \[
    \#C^{\mathrm{aff}}(\mathbb{F}_r)
    = \sum_{x \in \mathbb{F}_r}
    \Bigl(1 + \Bigl(\frac{f(x)}{r}\Bigr)\Bigr)
    = r + \sum_{x=0}^{r-1} \Bigl(\frac{f(x)}{r}\Bigr),
  \]
  where $(\cdot/r)$ is the Legendre symbol.  Thus
  $a_r = -\sum_{x=0}^{r-1} (f(x)/r)$.

  Since $f(-x) = -f(x)$ and $r \equiv 3 \pmod{4}$,
  we have $(-1/r) = (-1)^{(r-1)/2} = -1$.  Therefore
  \begin{align*}
    \Bigl(\frac{f(-x)}{r}\Bigr)
    &= \Bigl(\frac{-f(x)}{r}\Bigr)
    = \Bigl(\frac{-1}{r}\Bigr)
      \Bigl(\frac{f(x)}{r}\Bigr)
    = -\Bigl(\frac{f(x)}{r}\Bigr).
  \end{align*}
  Pairing $x$ with $-x$ in the sum (noting $f(0) = 0$
  contributes~$0$) gives
  \[
    a_r = -\sum_{x=0}^{r-1}\Bigl(\frac{f(x)}{r}\Bigr)
    = -\Bigl(\frac{f(0)}{r}\Bigr)
    - \sum_{x=1}^{(r-1)/2}
    \left[\Bigl(\frac{f(x)}{r}\Bigr)
    + \Bigl(\frac{f(-x)}{r}\Bigr)\right]
    = 0.\qedhere
  \]
\end{proof}

\begin{remark}\label{rem:r1}
  For $r \equiv 1 \pmod{4}$, we have $(-1/r) = +1$, so
  the terms reinforce rather than cancel.
  Computationally, $a_r = 0$ at $\sim\!50\%$ of primes
  $r \equiv 1 \pmod{4}$ (Table~\ref{tab:traces}).
\end{remark}

\begin{figure}[H]
  \centering
  \includegraphics[width=\textwidth]{../figures/fig_trace_vanishing.pdf}
  \caption{The trace vanishing law for $(p,q) = (3,7)$.
    (a)~Frobenius trace $a_r$ vs.\ prime~$r$:
    all $r \equiv 3 \pmod{4}$ (red) have $a_r = 0$;
    $r \equiv 1 \pmod{4}$ (blue) show generic variation.
    (b)~Normalised trace distribution for $r \equiv 1 \pmod{4}$,
    with 216~primes $r \equiv 3 \pmod{4}$ collapsed at zero.}
  \label{fig:trace}
\end{figure}


% ====================================================================
\section{Endomorphism Ring}\label{sec:endo}

\begin{proposition}\label{prop:endQ}
  For generic distinct odd primes $p \ne q$,
  $\End_{\mathbb{Q}}(\Jac(C)) = \mathbb{Z}$.
\end{proposition}

\begin{proof}[Proof sketch]
  Three potential sources of extra endomorphisms are excluded:

  \emph{Jacobian splitting.}
  By the Kani--Rosen criterion, $\Jac(C)$ splits over
  $\mathbb{Q}$ if and only if~$C$ admits a
  $\mathbb{Q}$-rational involution other than the
  hyperelliptic involution $\iota\colon (x,y) \mapsto (x,-y)$.
  Since $\Aut_{\mathbb{Q}}(C) = \langle\iota\rangle
  \cong \mathbb{Z}/2$ (the automorphism~$w$ requires $i$),
  no splitting involution exists over~$\mathbb{Q}$
  (cf.\ Cardona--Quer~\cite{CardonaQuer}).

  \emph{Real multiplication.}
  The polynomial $f(x) = x \cdot h(x^2)$ with
  $h(t) = t^2 - (p^2 + q^2)t + p^2 q^2$ has discriminant
  $\Delta_h = (p^2 - q^2)^2$, a perfect square.
  By Mestre's criterion~\cite{Mestre}, the RM field is
  $\mathbb{Q}(\sqrt{\Delta_h}) = \mathbb{Q}$, so RM
  degenerates to~$\mathbb{Z}$.

  \emph{Complex multiplication.}
  CM requires special algebraic relations between
  $p$ and~$q$, which do not hold for generic primes.
\end{proof}


% ====================================================================
\section{Sato--Tate Group}\label{sec:ST}

By the FKRS classification~\cite{FKRS}, the Sato--Tate group
lies in the ``$C_2$ family'':

\begin{proposition}\label{prop:ST}
  The Sato--Tate group of $\Jac(C)$ is contained in
  $N(\mathrm{U}(1) \times \mathrm{U}(1)) \subsetneq \USp(4)$.
\end{proposition}

The computational signatures are universal across the family:
$a_r = 0$ for all $r \equiv 3 \pmod{4}$ (proved), and
$a_r = 0$ for $\sim\!50\%$ of $r \equiv 1 \pmod{4}$.
Figure~\ref{fig:cross} confirms this across five test curves.

\begin{table}[H]
\centering
\begin{tabular}{rrrrl}
\toprule
$r$ & $r \bmod 4$ & $a_r$ & $t_r = a_r/2\sqrt{r}$ & \\
\midrule
11 & 3 &  0 & $0$ & vanishing law \\
13 & 1 &  0 & $0$ & \\
17 & 1 & $-4$ & $-0.485$ & nonzero \\
19 & 3 &  0 & $0$ & vanishing law \\
23 & 3 &  0 & $0$ & vanishing law \\
37 & 1 & $-4$ & $-0.329$ & nonzero \\
41 & 1 & $-4$ & $-0.312$ & nonzero \\
53 & 1 &  0 & $0$ & \\
89 & 1 & $28$ & $\phantom{-}1.484$ & nonzero \\
101 & 1 & $20$ & $\phantom{-}0.995$ & nonzero \\
\bottomrule
\end{tabular}
\caption{Frobenius traces for $(p,q) = (3,7)$.
  The normalised trace $t_r \in [-2, 2]$ by the Weil bound.}
\label{tab:traces}
\end{table}

\begin{figure}[H]
  \centering
  \includegraphics[width=\textwidth]{../figures/fig_cross_validate.pdf}
  \caption{%
    (a)~Percentage of good primes with $a_r = 0$, by residue
    class and curve.  All curves show $100\%$ vanishing at
    $r \equiv 3 \pmod{4}$ and $\sim\!50\%$ at
    $r \equiv 1 \pmod{4}$.
    (b)~Complete classification of local arithmetic.}
  \label{fig:cross}
\end{figure}


% ====================================================================
\section{Local $L$-Factors}\label{sec:local}

At bad odd primes, Paper~\cite{Chen2026UTS} identified two
reduction types.  The conductor exponent $f_r = 2$ is
unaffected by node splitting, but the local $L$-factors require
a finer analysis.

\subsection{Cases I/II: Cuspidal reduction ($r = p$ or $q$)}

When $r = p$, the roots $0, p, -p$ collide modulo~$r$,
and the reduced curve is
$\bar{C}\colon y^2 = x^3(x^2 - \bar{q}^{\,2})$.
Setting $v = y/x$ gives the normalisation
\begin{equation}\label{eq:Enorm}
  \tilde{E}\colon v^2 = x^3 - \bar{q}^{\,2}\, x
  \qquad (\text{over } \mathbb{F}_r),
\end{equation}
an elliptic curve of the form $v^2 = x^3 + ax$ with
$a = -\bar{q}^{\,2}$ and $b = 0$.  Its $j$-invariant is
$j = 1728$, so $\tilde{E}$ has complex multiplication
by~$\mathbb{Z}[i]$.  (Symmetrically, when $r = q$, the
normalisation is $v^2 = x^3 - \bar{p}^{\,2}\, x$,
again with $j = 1728$.)

The local $L$-factor is determined by the Frobenius
trace $a_E$ on~$\tilde{E}$:
\begin{equation}
  L_r(s) = (1 - a_E\, r^{-s} + r^{1-2s})^{-1}.
\end{equation}
Since $\tilde{E}$ has CM by $\mathbb{Z}[i]$, the
trace $a_E$ satisfies $a_E = 0$ when $r \equiv 3 \pmod{4}$
(by Deuring's theorem), reflecting the same residue-class
dichotomy as the trace vanishing law.

\subsection{Cases III/IV: Nodal reduction}

The reduced curve has two $A_1$ nodes at
$x = \bar{a}$ and $x = -\bar{a}$.  Each node is
\emph{split} (tangent slopes in $\mathbb{F}_r$, contributing
$(1 - r^{-s})^{-1}$) or \emph{non-split}
(slopes in $\mathbb{F}_{r^2}$, contributing
$(1 + r^{-s})^{-1}$), depending on the quadratic residue
character.

\begin{proposition}\label{prop:nodesplit}
  The node at $x = \bar{a}$ is split iff
  $(\bar{a}/r) = 1$.  The node at $x = -\bar{a}$ is split
  iff $(-\bar{a}/r) = 1$.
\end{proposition}

\begin{corollary}\label{cor:localfactor}
  If $r \equiv 3 \pmod{4}$: exactly one node is split, giving
  \begin{equation}\label{eq:mixed}
    L_r(s) = (1 - r^{-s})^{-1}(1 + r^{-s})^{-1}
    = (1 - r^{-2s})^{-1}.
  \end{equation}
  If $r \equiv 1 \pmod{4}$: both nodes have the same type,
  giving $(1 - r^{-s})^{-2}$ $($both split$)$ or
  $(1 + r^{-s})^{-2}$ $($both non-split$)$.
\end{corollary}

\begin{remark}
  The conductor exponent $f_r = 2$ depends only on the toric
  rank $t = 2$, not on the splitting type.  The formula
  of Paper~\cite{Chen2026UTS} is unaffected.
\end{remark}

\begin{table}[H]
\centering
\begin{tabular}{llll}
\toprule
Case & Condition & $r \bmod 4$ & Local $L$-factor \\
\midrule
I/II (cusp)
& $r = p$ or $q$
& any
& $(1 - a_E\, r^{-s} + r^{1-2s})^{-1}$ \\
\addlinespace
III/IV (nodes)
& $r \mid N$ or $r \mid (p{-}q)$
& $\equiv 3$
& $(1 - r^{-2s})^{-1}$ \\
\addlinespace
III/IV (nodes)
& $r \mid N$ or $r \mid (p{-}q)$
& $\equiv 1$
& $(1 \mp r^{-s})^{-2}$ \\
\bottomrule
\end{tabular}
\caption{Refined local $L$-factors at bad odd primes.
  Cases~I/II have $j(\tilde{E}) = 1728$ (CM by $\mathbb{Z}[i]$).
  In Cases~III/IV with $r \equiv 1$, the sign depends
  on~$(\bar{a}/r)$.}
\label{tab:local}
\end{table}


% ====================================================================
\section{Weil Restriction and the Asai $L$-Function}
\label{sec:weil}

\subsection{The Galois representation is induced}

The involution $w^*$ acts on $V_\ell = H^1(C_{\Qbar}, \Ql)$
with eigenvalues $\pm i$.  Write $V_\ell = V^+ \oplus V^-$
for the eigenspaces.  Complex conjugation swaps
$V^+$ and~$V^-$, so
\begin{equation}\label{eq:induced}
  V_\ell \;\cong\;
  \mathrm{Ind}_{G_{\mathbb{Q}(i)}}^{G_{\mathbb{Q}}}(V^+).
\end{equation}
By Mackey's formula, the trace of $\sigma_r$ on $V_\ell$
vanishes at inert primes ($r \equiv 3 \pmod{4}$),
confirming Theorem~\ref{thm:vanish} representation-theoretically.

\subsection{Weil restriction}

\begin{proposition}\label{prop:weil}
  Over~$\mathbb{Q}(i)$, the Jacobian $\Jac(C)$ is
  $(2,2)$-isogenous to $E_1 \times E_2$, where
  \begin{align}\label{eq:E1E2}
    E_1 &\colon Y^2 = X(X - p^2)(X - q^2), \\
    E_2 &\colon Y^2 = X(X + p^2)(X + q^2). \notag
  \end{align}
  Both $E_1$ and $E_2$ are individually defined
  over~$\mathbb{Q}$, but the $(2,2)$-isogeny
  $\varphi\colon \Jac(C)_{/\mathbb{Q}(i)} \to E_1 \times E_2$
  is defined only over $\mathbb{Q}(i)$: complex conjugation
  acts on~$\varphi$ by swapping the two factors.
  Over~$\mathbb{Q}$, $\Jac(C)$ is therefore isogenous to
  $\Res_{\mathbb{Q}(i)/\mathbb{Q}}(E)$, where
  $E/\mathbb{Q}(i)$ is the elliptic curve whose $\ell$-adic
  representation is the eigenspace~$V^+$.
\end{proposition}

\begin{proof}[Construction]
  The isogeny arises from the Richelot construction
  associated to the partition
  $\{0, \infty\}, \{p, -p\}, \{q, -q\}$ of the
  Weierstrass points, combined with the eigendecomposition
  of $w^*$ on $\Jac(C)_{/\mathbb{Q}(i)}$ via the
  idempotents $e^\pm = (1 \mp i\, w^*)/2$
  in $\End(\Jac(C)) \otimes \mathbb{Q}(i)$.
  The quadratics $g_1(x) = x^2 - p^2$ and
  $g_2(x) = x^2 - q^2$ yield the two elliptic factors
  $E_1\colon Y^2 = X(X - p^2)(X - q^2)$ and
  $E_2\colon Y^2 = X(X + p^2)(X + q^2)$.
\end{proof}

\begin{remark}\label{rem:twist}
  The curve $E_2$ is the quadratic twist of~$E_1$ by~$-1$:
  the substitution $X \mapsto -X$ in~$E_2$ gives
  $Y^2 = -X(-X+p^2)(-X+q^2) = -[X(X-p^2)(X-q^2)]$,
  so $E_2 \cong E_1^{(-1)}$.  Over $\mathbb{Q}(i)$,
  $-1 = i^2$ is a square, so $E_1$ and~$E_2$ become
  isomorphic---precisely the condition enabling the
  Richelot isogeny.  This $E \times E^{(-1)}$ structure
  is the geometric hallmark of the Asai lift: the
  degree-$4$ $L$-function is $L(E_1, s) \cdot L(E_1^{(-1)}, s)$
  over~$\mathbb{Q}$, which unifies into
  $L(E_1, s)^2$ over~$\mathbb{Q}(i)$.
  For $(p, q) = (3, 7)$: $E_1\colon Y^2 = X(X-9)(X-49)$
  and $E_2 = E_1^{(-1)}\colon Y^2 = X(X+9)(X+49)$.
\end{remark}

\subsection{The Asai $L$-function}

The $L$-function of $\Jac(C)/\mathbb{Q}$ is the
Asai $L$-function of $\pi_E$:
\begin{equation}\label{eq:asai}
  L(\Jac(C)/\mathbb{Q},\, s)
  \;=\; L^{\mathrm{As}}(\pi_E,\, s).
\end{equation}
Over $\mathbb{Q}(i)$, this factors as
$L(E, s) \cdot L(E^c, s)$, where $E^c$ is the
Galois conjugate.

The associated Siegel modular form is an
\emph{endoscopic lift} (Asai transfer from
$\GL_2(\mathbb{Q}(i))$ to $\GSp_4(\mathbb{Q})$),
\emph{not} a non-lift form.  The trace vanishing law
is precisely the signature of this endoscopic structure.


% ====================================================================
\section{Modularity}\label{sec:modularity}

The Weil restriction structure yields modularity without
invoking BCGP:

\begin{theorem}\label{thm:modularity}
  The abelian surface $\Jac(C)/\mathbb{Q}$ is modular:
  there exists a weight-$2$ Siegel paramodular form~$F$
  with $L(\Jac(C), s) = L(F, s)$, obtained by Asai transfer.
\end{theorem}

\begin{proof}[Proof sketch]
  By Allen--Calegari--Caraiani et al.~\cite{Allen},
  $E/\mathbb{Q}(i)$ is modular.  Automorphic induction from
  $\GL_2(\mathbb{Q}(i))$ to $\GL_4(\mathbb{Q})$ produces
  a cuspidal representation; the symplectic constraint on
  the Tate module transfers this to $\GSp_4(\mathbb{Q})$
  via Arthur's classification, yielding~$F$.
\end{proof}

\begin{remark}[BCGP does not apply]
  The BCGP theorem~\cite{BCGP} requires the mod-$\ell$
  Galois image to be ``vast,'' forcing
  $\mathrm{ST} = \USp(4)$.  Since our representation is
  induced, its image lies in a proper subgroup of
  $\GSp(4, \mathbb{F}_\ell)$ and is never vast.
  The automorphic induction route is both simpler and
  unconditional.
\end{remark}

\begin{remark}[Paramodular level and even conductor]
  The level is
  $N_A = 2^{f_2} \cdot
  [\rad_{\mathrm{odd}}(pqN(p{-}q))]^2$ with $f_2 \ge 4$.
  Since $N_A$ is even, the original Brumer--Kramer
  conjecture~\cite{BrumerKramer} (formulated for odd
  conductor) does not directly apply; the appropriate
  framework is Roberts--Schmidt~\cite{RobertsSchmidt}.
  While the existence of the automorphic representation~$\Pi$
  on $\GSp_4$ is guaranteed by the Asai transfer, predicting
  its exact local newform type at the wildly ramified prime
  $r = 2$ (and hence the exact paramodular level~$N_A$)
  remains conjectural and relies on the generalised
  Brumer--Kramer/Roberts--Schmidt framework.
  The odd part of the conductor is rigorously established
  by Papers~\cite{Chen2026TCV,Chen2026UTS}.
\end{remark}


% ====================================================================
\section{Discussion}

The Goldbach constraint $p + q = 2N$ forces the Jacobians to
be Weil restrictions of elliptic curves over~$\mathbb{Q}(i)$:
\[
  p + q = 2N
  \;\Longrightarrow\;
  f(-x) = -f(x)
  \;\Longrightarrow\;
  V_\ell = \mathrm{Ind}_{G_{\mathbb{Q}(i)}}^{G_{\mathbb{Q}}}(V^+)
  \;\Longrightarrow\;
  \Jac(C) \sim \Res_{\mathbb{Q}(i)/\mathbb{Q}}(E).
\]
Modularity reduces to that of $E/\mathbb{Q}(i)$, which is
known.  The local description is now complete at every prime:
conductor exponent (Papers~\#12--13), reduction type,
split/non-split refinement (Corollary~\ref{cor:localfactor}),
and the Asai structure of the $L$-function.

The reduction to elliptic curves over $\mathbb{Q}(i)$
suggests that Goldbach-type questions might be
approachable through the arithmetic of Bianchi modular
forms---a direction that merits further investigation.


% ====================================================================
\section*{Acknowledgments}

The author thanks the anonymous reviewers whose critiques
identified the Weil restriction structure and the Asai lift,
leading to fundamental improvements.
Scripts and data are at
\url{https://github.com/Ruqing1963/goldbach-frey-weil-restriction}.


\begin{thebibliography}{99}

\bibitem{Chen2026TCV}
R.~Chen,
\emph{The true conductor of Goldbach--Frey curves},
Zenodo, 2026.  \url{https://zenodo.org/records/18749731}

\bibitem{Chen2026UTS}
R.~Chen,
\emph{Universal tame semistability of Goldbach--Frey Jacobians},
Zenodo, 2026.  \url{https://zenodo.org/records/18751169}

\bibitem{Chen2026Census}
R.~Chen,
\emph{A conductor census of 425,082 Goldbach--Frey curves},
Zenodo, 2026.  \url{https://zenodo.org/records/18751442}

\bibitem{BrumerKramer}
A.~Brumer and K.~Kramer,
Paramodular abelian varieties of odd conductor,
\emph{Trans.\ Amer.\ Math.\ Soc.}~\textbf{366} (2014),
2463--2516.

\bibitem{RobertsSchmidt}
B.~Roberts and R.~Schmidt,
\emph{Local Newforms for $\GSp(4)$},
Lecture Notes in Math.~\textbf{1918}, Springer, 2007.

\bibitem{BCGP}
G.~Boxer, F.~Calegari, T.~Gee, and V.~Pilloni,
Abelian surfaces over totally real fields are potentially modular,
\emph{Publ.\ Math.\ IHES}~\textbf{134} (2021), 153--501.

\bibitem{Allen}
P.\,B.~Allen et al.,
Potential automorphy over CM fields,
\emph{Ann.\ of Math.}~(2), to appear.  arXiv:1812.09999.

\bibitem{FKRS}
F.~Fit\'e, K.\,S.~Kedlaya, V.~Rotger, and A.\,V.~Sutherland,
Sato--Tate distributions and Galois images,
\emph{Compos.\ Math.}~\textbf{148} (2012), 1390--1442.

\bibitem{PoorYuen}
C.~Poor and D.\,S.~Yuen,
Paramodular cusp forms,
\emph{Math.\ Comp.}~\textbf{84} (2015), 1401--1438.

\bibitem{Serre1959}
J.-P.~Serre,
\emph{Groupes alg\'ebriques et corps de classes},
Hermann, 1959.

\bibitem{CardonaQuer}
G.~Cardona and J.~Quer,
Field of moduli and field of definition for curves of genus~2,
World Scientific, 2005, pp.~71--83.

\bibitem{Mestre}
J.-F.~Mestre,
Construction de courbes de genre~2 \`a partir de leurs modules,
Birkh\"auser, 1991, pp.~313--334.

\end{thebibliography}

\end{document}
